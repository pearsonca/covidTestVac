\documentclass{article}
%\bibliographystyle{unsrtnat}

\usepackage[nohead, nomarginpar, margin=1in, foot=.25in]{geometry}

\bibliographystyle{plain}
\usepackage[tbtags]{amsmath}
\usepackage{MnSymbol}
\usepackage{xparse}
\usepackage{tabularx}
\usepackage{multirow}
\usepackage[export]{adjustbox}
\usepackage{graphicx}
\usepackage{hyperref}

\usepackage{datetime}

\newdateformat{UKvardate}{%
\THEDAY\ \monthname[\THEMONTH], \THEYEAR}
\UKvardate

%\usepackage{svg}

\usepackage[raggedright]{titlesec}

\usepackage{xcolor}
\usepackage[printwatermark]{xwatermark}
\usepackage{tikz}

%\newsavebox\mybox
%\savebox\mybox{\tikz[color=black!30,opacity=0.3]\node[align=center]{DRAFT \\ Do not distribute};}
%\newwatermark*[
%  allpages,
%  angle=45,
%  scale=8,
%  xpos=-30,
%  ypos=35
%]{\usebox\mybox}

\usepackage[parfill]{parskip}
%\parindent 0pt
%\parskip 12pt

\renewcommand{\baselinestretch}{1.15}

% \newcommands{\name}[number of args]{command} here
\newcommand{\eref}[1]{Eq.~\ref{#1}}
\newcommand{\fref}[1]{Fig.~\ref{#1}}
\newcommand{\sref}[1]{Section~\ref{#1}}
\newcommand{\srefs}[1]{Sections~\ref{#1}}

\newcommand{\dPois}[1]{\sim \textrm{Pois}(#1)}

% \def\name{definition} here
\def\eg*{\textit{e.g.}}
\def\ie*{\textit{i.e.}}
\def\nb*{\textit{n.b.}}
\def\etc*{\textit{etc.}}

\def\TPR{\textrm{TPR}}
\def\TNR{\textrm{TNR}}
\def\FPR{\textrm{FPR}}
\def\FNR{\textrm{FNR}}

\def\PPD{\textrm{PPD}}
\def\CPP{\textrm{CPP}}
\def\CPD{\textrm{CPD}}

\def\rhop{\rho_{{+}}}
\def\rhon{\rho_{{-}}}

\newcommand{\abs}[1]{\lvert #1\rvert}
\renewcommand{\thesection}{S\arabic{section}}
\renewcommand{\thetable}{S\arabic{table}}
\renewcommand{\thefigure}{S\arabic{figure}}
\renewcommand{\theequation}{S\arabic{equation}}

\usepackage{lineno}
\linenumbers

\begin{document}

\def\partitle{Pre-vaccination testing could expand coverage of 2-dose COVID vaccines}

\author[1,2,*]{Carl A. B. Pearson}
\author[1]{Sam Clifford}
\author[2]{Juliet R. C. Pulliam}
\author[1]{\mbox{Rosalind M. Eggo}}
\affil[1]{Department of Infectious Disease Epidemiology \&\ Centre for Mathematical Modelling of Infectious Diseases,
London School of Hygiene \&\ Tropical Medicine

Keppel Street, London, United Kingdom WC1E 7HT

\{carl.pearson, sam.clifford, r.eggo\}@lshtm.ac.uk}
\affil[2]{DSI-NRF Centre of Excellence in Epidemiological Modelling and Analysis, Stellenbosch University

19 Jonkershoek Road, Stellenbosch, South Africa, 7600

pulliam@sun.ac.za
}
\affil[*]{corresponding}

\title{Supplement for \partitle}

\maketitle

%\clearpage

%\tableofcontents
%\listoftables
%\listoffigures

%\clearpage

\section{Overview}

Assuming the availability of a sufficiently affordable and accurate point-of-care antibody test, and confirmation of initial laboratory-based evidence that SARS-CoV-2 antibody positive individuals (\ie* seropositive individuals) receive as much protection from a single dose as seronegative individuals receive from two doses, a test-and-vaccinate programme could be effective at expanding vaccine coverage and reducing per-protected-individual costs.

In this supplement, we provide the mathematical details of a direct benefit analysis of a homogeneous population receiving the vaccine. This simplifies elements that would increase the benefit of testing (\eg* no indirect benefits due to transmission reduction), but also ignores aspects that could reduce those benefits (\eg* same seroprevalence in prioritized populations). We also assume that the target population is not potentially subject to saturate effects; \eg* we can ignore how to deal with doubling 60\% coverage.

\section{Expanded Protection}

We assume homogeneous distribution of detectable prior SARS-CoV-2 infection that can be characterized by a population level seroprevalance, $\rhop = 1 - \rhon$. Similarly, we assume that the test can be characterized uniformly by two parameters, sensitivity (the true-positive rate, $\TPR = 1 - \FNR$), and specificity (the true-negative rate, $\TNR = 1 - \FPR$).

If we assume pessimistically that a single dose is not protective when given to seronegative individuals, then the people protected per two doses ($\PPD$) is:

$$
\PPD = 1\left(\rhon * \TNR + \rhop * \FNR \right) + 2\rhop*\TPR + 0\left(\rhon * \FPR\right)
$$

which we can re-arrange in terms of complementary parameters:

\begin{equation}\label{ppd}
\begin{aligned}
\PPD &= \left((1-\rhop) * \TNR + \rhop * (1-\TPR) \right) + 2\rhop*\TPR \\
&= \TNR-\rhop\TNR + \rhop - \TPR\rhop + 2\rhop*\TPR \\
&= \TNR - \rhop\TNR + \rhop + \TPR\rhop \\
&= \TNR + \rhop \left(1 + \TPR-\TNR\right)
\end{aligned}
\end{equation}

with the percent change in $\PPD$ being

$$
\Delta_{\PPD} = \TNR + \rhop \left(1 + \TPR-\TNR\right) - 1
$$

Note that given the assumption that a single dose is non-protective in seronegative individuals, this scheme can reduce the number of people effectively vaccinated. As practical matter, this is only a problem with extreme combinations of seroprevalence and test performance, which are practically irrelevant for the settings where this scheme is worth considering. Also, no benefit for a single dose also seems conservative. However, under extremely poor specificity \eg*:

$$
\lim_{\TNR\rightarrow 0}\Delta_{\PPD} = \rhop \left(1 + \TPR\right) - 1 = \rhop(1+\TPR) - 1 
$$

meaning $\rhop > 0.5$ required to provide benefit, and even higher for imperfect sensitivity.

\section{Cost}

Consider total cost per vaccine dose of $V$ (\ie* production, logistics, and administration) and total cost per test of $T$. Under a testing program, the cost of administering a dose is either $T+V$ (first dose) or $V$ (second dose). For every first dose administered, the scheme also administers second doses to individuals that test negative (accurately or not). The probability of that second dose is $\rhon\TNR + \rhop(1-\TPR)$, meaning that for every first dose there are that many second doses, and therefore the fraction of all doses that are first doses is:

$$
\frac{1}{1+(1-\rhop)\TNR + \rhop(1-\TPR)} = 
$$

So the expected cost per dose (denominated in the price of the vaccine, $V$) is

$$
\CPD = 1 + \frac{T}{V}\left(1+(1-\rhop)\TNR + \rhop(1-\TPR)\right)^{-1}
$$

and the cost per person protected is $\CPP = \frac{2\CPD}{\PPD}$. Relative to the cost of a no-test scheme cost per person protected of 2 (\nb* because we are denominating in the price of the vaccine $V$), the change in cost per person protected is:

$$
\Delta_{\CPP} = \frac{\CPD}{\PPD}-1 =\frac{1 + \frac{T}{V}\left(1+(1-\rhop)\TNR + \rhop(1-\TPR)\right)^{-1}}{\TNR + \rhop \left(1 + \TPR-\TNR\right)} - 1
$$

\section{General Results}



%\bibliography{refs}

\end{document}

% 